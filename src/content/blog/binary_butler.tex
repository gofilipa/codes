% Created 2025-02-10 Mon 11:34
% Intended LaTeX compiler: pdflatex
\documentclass[11pt]{article}
\usepackage[utf8]{inputenc}
\usepackage[T1]{fontenc}
\usepackage{graphicx}
\usepackage{longtable}
\usepackage{wrapfig}
\usepackage{rotating}
\usepackage[normalem]{ulem}
\usepackage{amsmath}
\usepackage{amssymb}
\usepackage{capt-of}
\usepackage{hyperref}
\author{fcalado}
\date{\today}
\title{}
\hypersetup{
 pdfauthor={fcalado},
 pdftitle={},
 pdfkeywords={},
 pdfsubject={},
 pdfcreator={Emacs 29.3 (Org mode 9.6.15)}, 
 pdflang={English}}
\begin{document}

\tableofcontents

---
title: (Gender) Binary in the Machine
description: 'What is the role of the gender binary in machine
learning processes?'
pubDate: 'January 30, 2025'
heroImage: '/codes/binary\_final.png'
---

How do machine learning processes

Gender binaries

I often get this question about binaries when I'm talking about
resisting gender binaries within a technological context: but aren't
machines built on binary structures? Isn't the technology, at its
core, a bunch of zeroes and ones anyway? The question is convincing.
It suggests that if machine systems are founded on a particular, are
made possible through that schema, then there is something
foundationally true about that schema. Of course the person who is
asking the question never quite phrases it in this way, but I imagine
that they are thinking it. Or at least I am thinking it, which is
enough anyway.

The role of the binary as a constituting but not for that reason
finishing structure can be traced back to the godparent of Queer
Studies, Judith Butler. 


signifiers produce " a full and final recognition that can never be
achieved." (143)

"The practice by which gendering occurs, the embodying of norms, is a
compulsory practice, a forcible production, but not for that reason
fullydetermining. To the extent that gender is an assignment, it is an
assignment which is never quite carried out according to expectation,
whose addressee never quite inhabits the ideal s/he is compelled to
approximate." (180)


"Binary code" by Christiaan Colen is licensed under CC BY-SA 2.0;
overlay,
[combotrans.svg](\url{http://creativecommons.org/licenses/by-sa/3.0/}), by
user:pschemp, CC BY-SA 3.0, via Wikimedia Commons. 
\end{document}
