% Created 2025-12-04 Thu 16:21
% Intended LaTeX compiler: pdflatex
\documentclass[11pt]{article}
\usepackage[utf8]{inputenc}
\usepackage[T1]{fontenc}
\usepackage{graphicx}
\usepackage{longtable}
\usepackage{wrapfig}
\usepackage{rotating}
\usepackage[normalem]{ulem}
\usepackage{amsmath}
\usepackage{amssymb}
\usepackage{capt-of}
\usepackage{hyperref}
\author{fcalado}
\date{\today}
\title{}
\hypersetup{
 pdfauthor={fcalado},
 pdftitle={},
 pdfkeywords={},
 pdfsubject={},
 pdfcreator={Emacs 29.3 (Org mode 9.6.15)}, 
 pdflang={English}}
\begin{document}

\tableofcontents

\section{Counter-Code: Computational Methods for Working with Language}
\label{sec:org09c2b3c}
\subsection{topic:}
\label{sec:orgf7a8a9d}
\begin{itemize}
\item Computational methods for working with language about queerness.
\end{itemize}

\subsection{book questions:}
\label{sec:org91f3b34}
\begin{enumerate}
\item What do concepts from Queer Studies and adjacent fields offer to
computational tools for working with text?
\item How do computational methods constrain the expressive potential of
language into data?
\item How do concepts from Queer Studies and adjacent fields work against
constraints in language data?
\end{enumerate}

Template answers to questions:

\begin{enumerate}
\item What do concepts from Queer Studies and adjacent fields offer to
computational tools for working with text?
\begin{itemize}
\item\relax [queer theory concept], offers an approach to [computational
tool] that [does this to text].
\begin{enumerate}
\item The Queer Studies concept of Gender Performativity offers an
approach to text analysis that surfaces multiplicity in
language meaning.
\item QOCC's concept of dis-identification offers an approach to
textual markup that delineates irreconcilable interpretations
of language choices.
\item Black Feminist Studies' concept of the flesh offers an
approach for analyzing text display that is grounded in the
physicality and materiality of the screen.
\item Trans Studies' concept of normativity offers an approach to
text generation that reveals shared investments across
identity and subject groups.
\end{enumerate}
\end{itemize}

\item How do these computational tools constrain the expressive
potential of language into data?
\begin{itemize}
\item\relax [Constraint] [does something] to language [to make it this].
\begin{enumerate}
\item In text analysis, "loops" strip language of idiosyncracies in
order to count its common features.
\item In textual markup, "tags" disambiguate and fix language to
sort it into discrete elements within an ordered hierarchy.
\item In text display, layers of computer software displace language
processing away from human awareness in order to streamline
interface effects.
\item In text generation, algorithmic prediction aggregates language
expressions to generalize and approximate patterns of language
use.
\end{enumerate}
\end{itemize}

\item How do concepts from Queer Studies and adjacent fields work against
constraints in language data?
\begin{itemize}
\item Queer studies [concept] [does a thing to work within and against
constraint]
\begin{enumerate}
\item From the concept of iteration, the emphasis on repetition
influences a text analysis practice that multiplies and
re-signifies word meaning.
\item From the concept of dis-identification, the appropriation of
dominant codes influences a markup strategy that delineates
the boundaries of tagging structures.
\item From the concept of the flesh, the foreclosure of depth
suggests a close reading strategy that unleashes potential
screen effects.
\item From the concept of normativity, the generalization of
language suggests a focus on shared investments between
polarized perspectives.
\end{enumerate}
\end{itemize}
\end{enumerate}
\subsection{book argument}
\label{sec:orga339369}
Concepts from Queer Studies and adjacent fields offer methodological
approaches to computational tools that work with text-based data. Such
tools, which are used for tasks like text generation, preservation,
and display, tend to constrain the expressive potential of language,
stripping its idiosyncrasies and ambiguities, to turn it into
computable units of data. Queer Studies and adjacent fields inspire
methodologies for working against these constraints---methodologies
that destabilize authoritative and delimiting structures, and open
novel readings of language data. By engaging closely and attentively
with these computational processes, users can resist the reductive
ways that technology works upon them.

\subsection{evidence}
\label{sec:org7ee5c72}
\subsubsection{main examples or case studies:}
\label{sec:org8ce65b8}
\begin{itemize}
\item Computational tools and underlying processes.
\item Theoretical concepts from minoritarian fields, queer theory mostly.
\item Literary sources.
\item Critical work in DH and Media Studies
\end{itemize}

\subsubsection{chapter by chapter evidence base}
\label{sec:org7950e4a}
\begin{itemize}
\item Chapter 1: Refers to text analysis methods and processes, the loop
in python programming, word vectors, Queer Theory's concept of
Gender Performativity, V Woolf's Orlando.
\item Chapter 2: Refers to text encoding practices and projects, the XML
data structure, Queer Theory's concept of dis-identification, Oscar
Wilde's MS for The Picture of Dorian Gray.
\item Chapter 3: Refers to media archaeological theory and practices,
Black feminist studies theory, software layers and abstraction,
Octavia Butler's Dawn and Entropy8Zuper's skinonskinonskin.
\item Chapter 4: Refers to ML/text generation methods and practices, ML
processes of training, Trans Studies theory on normativity and
embodiment, popular discourse on transphobia and Reality TV.
\end{itemize}

\subsection{chapter snapshots}
\label{sec:org247c457}
\begin{itemize}
\item Chapter 1:
\begin{itemize}
\item topic: text analysis to study gender
\item scope and evidence type:
\begin{itemize}
\item Computational text analysis algorithms like Logistic Regression,
Burrow's Delta.
\item the Queer Studies Theory of Gender Performativity,
\item Woolf's \emph{Orlando}, Rosenberg's \emph{Confessions of the Fox} or
\emph{Nightwood} or \emph{Fun Home}.
\end{itemize}
\item main claims:
\begin{itemize}
\item That dominant methods in text analysis use algorithms in a way
that close off the kinds of readings we can peform on gender.
\item Python processes text to diminish peculiarities and move toward
abstract connections.
\item One can iterate over binary structures in algorithms to open up,
rather than collapse, possible associations in gender terms.
\end{itemize}
\end{itemize}
\item Chapter 2:
\begin{itemize}
\item topic: text preservation/markup to study sexuality
\item scope and evidence type:
\begin{itemize}
\item Work and theory in textual scholarship and queer historiography.
\item TEI technology and projects that attempt to markup complex
sexual IDs,
\item The first 3 chapters of the manuscript of Dorian Gray, maybe
\emph{Nightwood}, \emph{Confessions}.
\end{itemize}
\item main claims:
\begin{itemize}
\item TEI imposes not just disambiguation between elements but also a
power structure which delimits what can and cannot be said.
\item The hierarchical nature of the TEI imposes a top-down control
over all child elements; a situation that multiplicity will not
solve.
\item We cannot mark up what has been lost, but we can see our own
complicity in power structures.
\end{itemize}
\end{itemize}
\item Chapter 3:
\begin{itemize}
\item topic: text display to study sex
\item scope and evidence type:
\begin{itemize}
\item Media archaeology theorizing on materiality.
\item Black feminist theorizing on the flesh.
\item The novel Dawn by Octavia Butler and \emph{skinonskinonskin}
\end{itemize}
\item main claims:
\begin{itemize}
\item That the collision/indeterminacy between thinking and feeling is
a queer method of resisting circumscription within power
structures.
\item Across layers of display and source code, surface effects are
unmappable, fugitive, displaced, enabling significatory
possbilities that resist technological constraints.
\item The pairing down to surface enables connectivity, intimacy.
\end{itemize}
\end{itemize}
\item Chapter 4:
\begin{itemize}
\item topic: text generation to study embodiment
\item scope and evidence type:
\begin{itemize}
\item Trans studies on normativity
\item Machine learning technology and training process
\item Popular discourse in online journalism and reality tv
\end{itemize}
\item main claims:
\begin{itemize}
\item Machine learning is a tool of approximation, reducing meaning to
common denominators and flattening perspectives to univocality.
\item It can be used to find similar investments across polarizing
perspectives.
\end{itemize}
\end{itemize}
\end{itemize}
\subsection{organizing principle}
\label{sec:org87f8b22}
Each chapter takes a different process for working with text,
identifies a computational constraint that reduces textual detail in
some way, and finds a way of working the constraint into a
close-reading practice.

\begin{itemize}
\item process for working with text -> constraint
\item text analysis, cleaning -> iteration (loops)
\item text preservation, encoding -> containment (tags)
\item text display, transformation -> abstraction (display)
\item text generation, normalization -> approximation (prediction algoirthms)

\item constraint -> what it does to data -> how to resist
\item iteration -> regularizes/homogenizes data -> repeat with difference
\item containment -> leaves out data -> narrate between gaps
\item abstraction -> displaces data -> close reading source code
\item approximation -> approximates data -> description?
\end{itemize}

\subsection{topic, claim, and metacommentary}
\label{sec:orgb74ada1}
Chapter 1:
\begin{itemize}
\item topic: Text analysis and gender
\item claim: Text analysis methods that reduce complexity and variability
of gender expression into binary forms can be "iterated" through
different vantage points (close and distant) to surface
multiplicity.
\item metacommentary: Words have inherent multiplicity, meaning is always
expanding beyond the binary. (But also, meaning is discursive,
disembodied--lacking material realities, whitewashed).
\end{itemize}

Chapter 2:
\begin{itemize}
\item topic: Text encoding and sexuality
\item claim: Hierarchical formats that categorize data can also suggest
implicit power structures that delimit what is sayable and
permissable.
\item metacommentary: Meaning is inherent in organization and structure,
controlling what can and cannot be signified. (Markup, inclusion
into systems, isn't enough).
\end{itemize}

Chapter 3:
\begin{itemize}
\item topic: Text display and sex
\item claim: Materiality exists even in abstracted forms, and points to
significatory possibilities that resist circumscription.
\item metacommentary: HOWEVER, the materiality of the sign offers a way of
working within reductive signification processes.
\end{itemize}

Chapter 4:
\begin{itemize}
\item topic: Text generation and embodiment
\item claim: Approximation of meaning finds commonalities between
polarized perspectives.
\item metacommentary: HOWEVER, univocality is a way of finding shared
investments.
\end{itemize}

\subsection{{\bfseries\sffamily TODO} model books:}
\label{sec:org5672145}
\begin{enumerate}
\item heavy processing
\label{sec:org28f3fdd}
\item queer data studies?
\label{sec:org7f7d39a}
\item \href{https://mitpress.mit.edu/9780262037006/the-fabric-of-interface/}{The Fabric of Interface Mobile Media, Design, and Gender}, Stephen Monteiro, 2017
\label{sec:orge7c2799}
Topic: Contemporary digital media's connection to textile
arts/production. 

Claim: our everyday digital practice has taken on traits common to
textile and needlecraft culture, and textile metaphors used to
describe computing (weaving code, threaded discussions, zipped files,
software patches, switch fabrics) represent deeper connections between
digital communication and “homecraft” or “women's work.”

Evidence base:
\begin{itemize}
\item the production of hardware and software from their conception in the
nineteenth century through twentieth-first-century globalized
electronics industries
\item textual and visual discourses around the digital, interactive,
integrative practices of contemporary digital devices and
their networked users
\item needlecraft and textile techniques, handloom or quilting frame,
mechanics of sewing and needlework
\item augmented reality (AR) and VR systems such as Google Glass, Google
Cardboard, and Oculus Rift
\end{itemize}

It's a good fit. Considering both everyday digital practices and
textile practices. Drawing analogies between these. 

\item \href{https://mitpress.mit.edu/9780262517409/mechanisms/}{Mechanisms New Media and the Forensic Imagination}, Matthew G. Kirschenbaum, 2012
\label{sec:orgea93583}
\item \href{https://direct.mit.edu/books/monograph/5133/Image-ObjectsAn-Archaeology-of-Computer-Graphics}{Image Objects: An Archaeology of Computer Graphics}, Jacob Gaboury. 2023
\label{sec:org96d14e4}
The MIT Press DOI: \url{https://doi.org/10.7551/mitpress/11077.001.0001}
\item \href{https://www.upress.umn.edu/9781517918774/a-users-guide-to-the-age-of-tech/}{A User's Guide to the Age of Tech}, Grant Wythoff, 2025
\label{sec:org309e1bb}
U of Minnesota Press
\end{enumerate}
\subsection{book narrative outline}
\label{sec:org594eed2}
Paragraph 1: What does my book as a whole do?
\begin{itemize}
\item[{$\boxtimes$}] my book's topic: Computational methods for working with language
about queerness.
\item[{$\boxtimes$}] book argument (see above)
\item[{$\boxtimes$}] implicit lesson:
\begin{itemize}
\item By engaging closely and sensually with a technology, we can find
ways of resisting its reductions, the ways it works upon us.
\item Experimenting with how something works in depth and in a sensual
way will liberate us from being used and being users
\end{itemize}
\item[{$\boxtimes$}] To answer this question, I apply concepts from QS and related
fields to work against constraints in computation using examples
from literature as my test case. [or see evidence base above].
\end{itemize}

Paragraph 2: How does this book as a whole proceed? (how do the
arguments unfold over the chapters)
\begin{itemize}
\item[{$\boxtimes$}] organizing principle (see above): Each chapter takes a different
process for working with text, identifies a computational constraint
within that process, and finds a way of working that constraint into
a close-reading practice.
\item[{$\boxtimes$}] narrative arc (chapter 8)
\begin{itemize}
\item moving from simple concepts to the complex processes.
\begin{itemize}
\item introducing humanists to how computers work on a fundamental
level. We start with the simple, with counting and categorizing,
then we get deeper into stacks and machine learning.
\item the first two chapters are about simpler processes, counting and
tagging, and the latter chapters are about more complex
processes, abstraction and prediction.
\begin{itemize}
\item I see other parallels between the two parts: the first part
being about creating destabilization, and the second part
about re-stabilizing; the first about queer theory and its
methods, the second about adjacent fields and it's methods.
\end{itemize}
\end{itemize}
\end{itemize}
\item[{$\boxtimes$}] describe book structure (chapter 11):
\begin{itemize}
\item The book is divided into two parts: part one, which explores core
concepts in computation: chapter 1 explores counting, which is the
bedrock of all analysis tasks; chapter 2 explores tagging which is
the foundation of data formatting and structure. Then, part two
explores more complex processes: chapter 3 explores
interface/display processes, and chapter 4 explores text
generation. The second half builds on concepts from the first to
show more complex tools. Word vectors (ch 1) -> neural net
training (ch 4), markup languages (ch 2)- > html code interacting
with javascript (ch 3).
\end{itemize}
\end{itemize}

Paragraphs for body chapters:
\begin{itemize}
\item[{$\square$}] what does this chapter do in service of the book?
\item[{$\square$}] start with a topic sentence from "what stays the same, what
changes": 
\begin{itemize}
\item from chapter one to chapter two: We move from text analysis to
text encoding, from counting to markup. They are both basic,
foundational acts of working with text data, but while analysis is
about regularizing data to be counted, markup is about appending
descriptive information to sort and structure data.
\begin{itemize}
\item Impact:
\begin{itemize}
\item Markup shows the implicit ways that data formats pre-
determine what kind of information can be contained within
them. This information has to do with power structures that
delimit what can and cannot be said. "Occluding whiteness".
\item It also brings to the surface (in another way) how QS has
looked over race and materializations. In the previous
chapter, we found that the trope of fluidity overlooks
material reality. In this chapter, we find that there is no
formalization that will recuperate what has been left out of
the structure. That the task must be otherwise.
\end{itemize}
\end{itemize}
\item from chapter two to chapter three: Looking at a more complex
process, which is abstraction and display of data. Moving into
Black Feminist studies, to find ways of addressing material
concerns. We return to the materiality of the word, a stabilizing
force.
\begin{itemize}
\item Impact:
\begin{itemize}
\item Now we can really "go deeper" than we did in previous
chapters, where we merely dipped our toes.
\item We are zooming out to look at how a larger process, of
abstraction, works on data. There is more complexity, we've
learned more about how computation works to get here,
regularizing data, plain text formatting, in particular, which
helps the reader understand data stacks and how they work.
\end{itemize}
\end{itemize}
\item from chapter three to chapter four: Looking at algorithmic
prediction and how that works to normalize data, similar to what
we discussed in the first chapter. Moving to Trans Studies, to
address underlying investments in normativity.
\begin{itemize}
\item Impact:
\begin{itemize}
\item The approximation of language revealing shared investments
across political poles. Like the previous chapter, something
shared, something solid.
\end{itemize}
\end{itemize}
\end{itemize}
\item[{$\square$}] chapter snapshots (see above) on scope, evidence, and claims.
\item[{$\square$}] book answers for each chapter (see above)
\end{itemize}

\subsection{{\bfseries\sffamily TODO} book narrative}
\label{sec:orgb7ec87b}
Queer Computing / Queerware / Counter-Code

Methods for Resisting Computational Constraints of Text-based Data; Re-thinking language tools; Queer Deviance and the Reimagining of
Language Technologies; Queer and Feminist Revisions of Language
Technologies; Close reading language technologies to destabilize
power. 

This book explores computational processes that work with text. It
applies concepts from humanities fields to push against the ways that
computational processes reduce and flatten the complexity of language
data. It shows how certain computational tasks, like text generation,
preservation, and display, tend to constrain the expressive potential
of language, stripping its idiosyncrasies and ambiguities, to turn it
into computable units of data. Humanities fields like Queer Studies,
Black Feminist Studies, and Trans Studies inspire methodologies for
working against these constraints---methodologies that destabilize
authoritative and delimiting structures, and open novel readings of
language data. In its experiment with these new methodologies, this
book uses examples of Queer literature as a test case, applying
methods to \emph{read} elements of literary form in computational
structures and processes. By engaging closely and attentively with
these computational processes, the reader discovers tools for
resisting the ways that technology works upon language, moving
simultaneously with and against the system which constrains meaning.

Each chapter of this book takes up a different computational process
for working with text, investigates how that processes handles
language data, and identifies a constraint---a mechanism that
transforms and reduces an expressive quality of language into data.
Then, with the help of Queer Studies and related fields, it re-works
the constraint into a new practice that resists this reduction of
language, and applies this practice to reading a sample of Queer
literature. The earlier chapters focus on basic computational
mechanisms, like counting and tagging language data, while the later
chapters move toward more complex processes of abstraction and
prediction. Chapter One considers text analysis methods, and focuses
on the foundational computational mechanism of counting data. Chapter
Two turns to text preservation methods, and focuses on "tagging" or
categorizing text data. Then, moving from basic concepts to more
complex processes, Chapters Three and Four consider abstraction and
prediction, respectively, which are responsible for text display and
generation. Computational components from the first half of the book,
like word vectors and markup languages, are expanded in the second
half to describe more advanced concepts, like neural nets and
animation processes.

Chapter One, "Counting Text," examines a foundational act of
computation---counting. It explores counting in the context of
quantitative text analysis to study gender in Virginia Woolf's novel,
\emph{Orlando: A Biography} (1928). It delves into the mechanics behind
quantitative text analysis, identifying the core concept of the
"loop," which repeats the same code execution to bits of data, one by
one. This mechanism, which removes textual idiosyncrasies and details,
flattens and reduces text data into a standardized, computable form.
Then, the chapter connects the loop mechanism in code to the theory of
Gender Performativity, first propounded by Judith Butler in 1990's
\emph{Gender Trouble: Feminism and the Subversion of Identity}, which is
widely cited as an inaugural text of Queer Studies. From Butler's
formulation, it applies the iterativeness of gender performativity and
its potential for re-signifying terms to a text analysis pratice that
\emph{iterates} through close and distant reading. It performs this method
on a reading of gender in Woolf's novel, \emph{Orlando} (1928), to surface
the multiplicity inherent in language forms. Finally, it offers some
more recent critiques of this reading and of Gender Performativity for
the way it elides questions of embodiment and materiality.

Moving from counting text, Chapter Two, introduces "Tagging Text," a
procedure for organizing and sorting data into machine-readable
structures. This chapter applies the Text Encoding Initative (TEI), a
widely-used standard for tagging or "encoding" text-based data, to
mark the revisions of Oscar Wilde's manuscript of \emph{The Picture of
Dorian Gray} (1990), which was edited by the author to remove
references to homosexuality. This chapter explores the TEI's rigid
tagging structure, an organizational schematic made of the individual
tags that contains text elements within units that are nested within
one another. The hierarchical structure of the TEI imposes a top-down
control in which lesser or "child" elements are fixed and
disambiguated according to their "parent" element, a structure that
delimits what can and cannot be said within its boundaries. The
chapter then searches for methods for resisting this structure by
turning to Queer of Color Critique, and its concept of
dis-identification. This concept offers a method of re-appropriation,
"reading onesself into" structures of dominance. Applying this
analytic to the TEI allows one to delineate irreconcilable
interpretations of language choices in Wilde's revision process, an
experiment that surfaces not the content, but the contours, of what
cannot be recovered. The chapter ends with a meditation on Wilde's own
position of privilege as an editor of his own text, the executor of
his own censorship, and how TEI can be used as a way of surfacing
power operations in text.

The second half of the book moves from foundational processes in
computing like counting and tagging, to more complex processes of
layers of software abtraction and algorithmic prediction. Chapter
Three, "Text Display," examines how screen effects---displays and
animations---engage with data in its material form stored within the
computing "stack," consisting of layers of software. It takes an
example of early internet art, called "NET art", \emph{skinonskinonskin}
(1999), a series of "digital love letters" written by Aureia Harvey
and Michael Samyn (Rhizome, \emph{Net Art Anthology}). The chapter pursues
a close reading of the electronic work, examining how the form of
computer code and coding logics within its software layer bear on the
reading of the work's surface, the screen layer. To read the forms of
code, it draws from two seemingly unrelated fields, Media Archaeology
and Black Feminist Studies, both of which have much to say about how
physicality and materiality interact with abstract and symbolic
processes. Bringing these theorizations to a deep reading of
programming code reveals how screen effects, which seem disconnected
from inaccessible and obscure hardware processes, offer significatory
possbilities for reading materiality and physicality. As the net
artwork's layers of JavaScript and HTML code demonsrate, the paring
down to surface forms enables readings of connectivity and intimacy.

Building on the analysis of "Counting Text," or word frequencies, in
chapter one, the fourth and final chapter explores an advanced
counting processes, which is machine learning, which it uses to study
normativity. This chapter, "Text Generation," traces the steps behind
algorithmic prediction to examine what it does to language data and
how it represents word meaning in computational form. It posits the
the training process for Large Language Models (LLMs) as a tools that
\emph{normalizes} language, transforming word meaning into an approximate
or average forms. This process, which I examine in technical detail,
transforms word meaning into numerical scores, which it compiles from
averaging that word's various contexts from the training data. To
demonstrate this process, I train miniature LLMs from the transcripts
of a reality TV dating show, called \emph{Love Is Blind} (2020 - present).
In this show, the participants, who are all heterosexual and
cisgendered, are required to date each other from behind a wall,
without having access visual access to their partners until they have
agreed to get married. I then prompt the language model, which has
been trained to generate text in the same language and style of the
show participants, to answer questions about the role of the body and
embodied sensations in their dating experience. I use the language
model as a predictive instrument that can \emph{approximate} the linguistic
and affective patterns around the body and touch. Drawing on Trans
Studies theorizations of the body and of normativity, I argue that
participants in \emph{Love Is Blind}--—while firmly anchored in
cisnormative frameworks—--temporarily experience a form of bodily
dissonance. This dissonance, I argue, offers grounds for theorizing
connection and solidarity beyond identity politics and polarized
points of view.

The chapters in the second half of the book offers responses to the
problems spun from those in first half: from Chapter One, the problem
of discursive forms, and how they elide embodied experiences of
gender, and in Chapter Two, the problem of power structures, which are
totalizing. Chapters Four and Three, respectively, answer these
problems by grounding in what is shared and what is material. It shows
how Queer Studies, as a field, is good at offering certain
possibilities for resistance that \emph{disrupts} existing structures, for
performativity and dis-identification, but other fields, like Black
Feminist and Trans Studies, offer ways of resisting that are \emph{outside}
of the current structure. Both have advantages and disadvantages: both
are necessary for resistance work. 


ADD:
\begin{itemize}
\item queer concept to each chapter: gender, desire, sex, body
\item and this---not queer studies---is actually what gives me an
organizing principle around "queerness". Queer can mean Queer
Studies, but it also means things associated with gender, sexuality,
and intercourse. I am expanding the umbrella of queer, mostly
because we don't have another term that refers to all of these
simultaneously.
\end{itemize}
\end{document}
