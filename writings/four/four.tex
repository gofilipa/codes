% Created 2025-12-04 Thu 10:33
% Intended LaTeX compiler: pdflatex
\documentclass[11pt]{article}
\usepackage[utf8]{inputenc}
\usepackage[T1]{fontenc}
\usepackage{graphicx}
\usepackage{longtable}
\usepackage{wrapfig}
\usepackage{rotating}
\usepackage[normalem]{ulem}
\usepackage{amsmath}
\usepackage{amssymb}
\usepackage{capt-of}
\usepackage{hyperref}
\author{fcalado}
\date{\today}
\title{}
\hypersetup{
 pdfauthor={fcalado},
 pdftitle={},
 pdfkeywords={},
 pdfsubject={},
 pdfcreator={Emacs 29.3 (Org mode 9.6.15)}, 
 pdflang={English}}
\begin{document}

\tableofcontents

\section{"My Body Feels Like It's Coming Off": Cisgender Erotics In \emph{Love Is Blind}}
\label{sec:org475abbe}

\subsubsection{{\bfseries\sffamily TODO} ADD introduction: subjectivity \& reading practices}
\label{sec:org1a65e4a}
\begin{enumerate}
\item DRAFT sedgwick, knowledge --> method
\label{sec:org9dea166}
Over 25 years ago, in the midst of the first wave of the AIDS epidemic
and the government's neglect for those (mostly Black people and gay
men), whom it affected, Eve Kosofsky Sedgwick wonders about the point
of doing critique in the first place. Even if one could prove the the
government's indifference or disdain for people of color and for gays,
that "the lives of African Americans are worthless in the eyes of the
United States; that gay men and drug users are held cheap where they
aren't actively hated\ldots{} Supposing we were ever sure of all of those
things," Sedgwick writes, "what would we know then that we don't
already know?" (\emph{Novel Gazing} 3-4).

For me, this question catalyzes a shift in methodology around reading
practices, which shift from an orientation around knowledge ("Is a
particular piece of knowledge true?") toward method: "What does
knowledge do---the pursuit of it, the having and expressing of it?"
(\emph{Novel Gazing} 4). And this shift to method, first posed at the cusp
of the information age, is more urgent today. Not only is all of the
evidence (of neglect, of wrongdoing, of injustice by those in power)
at our fingertips; it is also self-generating and self-perpetuating in
the algorithmic processes that fill our feeds. We are distended with
it, all the proof, the polarization.

\item masculinity/femininity is
\label{sec:org4c2d305}
So I have sought a new reading method that responds to this moment, to
information overload and to automation. I trained some Large Language
Models (LLMs) deliberately to study it, trained them on text datasets
representing totally polarized perspectives.\footnote{Both of the models are openly licensed on Huggingface.co. See
Calado, \emph{gpt2-hertiage\_foundation-gender}, and Calado,
\emph{gpt2-aclu-gender}.} \textsuperscript{,}\,\footnote{For those who don't know, training models happens in various
stages. To train this model, I used a base model, which is already
trained, called gpt-2 (an open source model). I re-trained the base
model on the gender datasets. This is technically called
"fine-tuning", and its much easier and less resource intensive than
training the underlying base model.} I scraped
the data for training these models (a process explained in detail
below) from articles in the American Civil Liberties Union (ACLU) and
the Heritage Foundation.\footnote{The training data and code used to scrape the articles can be
found on github.com/gofilipa/anti-trans.} Then, I asked these biased models
questions about gender. At first, they responded in predictable ways.
For example, the ACLU model generated the following outputs about
"Masculinity" and "Femininity":
\begin{quote}
Masculinity is a matter of love and celebration.

Masculinity is a space for hope and liberation for all.

Masculinity is not defined solely by the beauty of our bodies, but by
the beauty of our experiences.

Femininity is a celebration of beauty, feminine liberation, and
femininity.

Femininity is our joy, our struggle, and our fight is our struggle.

Femininity is about allowing people to express themselves without
government interference.
\end{quote}
The outputs align with what one might expect from a perspective that
affirms gender diversity and expression. They frame gender as a
celebratory phenomenon, characterized by empowering language like
"liberation," "beauty", and "joy".

Meanwhile, in a similarly predictable way, the Heritage model,
associates "Masculinity" and "Femininity" with what are typically
conservative ideals based on tradition.
\begin{quote}
Masculinity is the cornerstone of Western civilization.

Masculinity is the fruit of patriarchy, and patriarchy is the heart
of conservatism.

Masculinity is defined by the ability to produce sperm, eggs, and live
children.

Femininity is an enduring American tradition.

Femininity is defined by means of the relationship between the sexes,
the ability to raise their children, the capacity to provide for their
own reproduction, the capacity to provide for their own children, the
ability to provide for their own.
\end{quote}
Like the ACLU model, these gender terms are positive. However, unlike
the ACLU, the terms here affix to notions of culture, tradition, and
reproduction—--things that suggest stability.

\item {\bfseries\sffamily TODO} ADD defintions of "gender" from exec order \& progressive side
\label{sec:org7c03df7}
However, there is something peculiar about these results from the
Heritage model, which centers on a particular, term, "subjectivity."
\begin{quote}
Masculinity is a subjective self-perception, not a universal
concept.

Femininity is a subjective, internal sense of self.

The gender binary is a subjective, malleable, and often incorrect
idea.

The gender binary is a subjective, internal, and often transitory
concept.

The gender binary is a subjective, grammatically incorrect and
illogical concept that conflates sex and gender identity.
\end{quote}
Unlike the other outputs, these examples do not describe the far-right
viewpoint on gender—--that it is based on the biological truth of two
sexes INSERT QUOTE FROM THE TRUMP EXECUTIVE ORDER. Rather, these
examples are closer to the progressive view of gender, which asserts
that gender describes identity, based on social behaviors, roles, and
expression, among other things. INSERT GENDER DEF FROM WHO(?) OR OTHER
PROG ORG.

\item {\bfseries\sffamily TODO} ADD amin \& butler on subjectivity
\label{sec:orge917eef}
The concept of "subjectivity" offers a nexus around which gender
suppressive perspective congregates. To them, it represents a serious
contradiction in the argument for gender rights. If gender is in fact
not rooted in biology, but a subjective, internal phenomenon, then why
would does \emph{the body} need to be changed to reflect it? Why is gender
transition necessary? CITE JUDITH BUTLER WHO IS AFRAID OF GENDER

Kadji Amin has offered an alternative to this thinking about gender.
Rather than a subjective, internal sense of self, gender is an
external expression, verifiable in presentation and behavior. CITE
AMIN ON GENDER BEING A PHYSICAL THING. So that, physical aspects of
the body matters the most in gender. Not only physical aspects of the
body, but the difference from the norm that these physical aspects
make. 

\item text generation as a reading method
\label{sec:org377d54a}
To counter "paranoid reading" practices, Sedgwick offers the method of
reparative reading. Reading reparatively is about reading
productively, for what associations a bit of text can generate.
Sedgwick takes "shame," an affect that is traditionally seen as
repressive, and examines how it creates productive and effects in
text. She describes shame as:
\begin{quote}
“a kind of free radical that (in different people and different
cultures) attaches to and permanently intensifies or alters the
meaning of—-of almost anything: a zone of the body, a sensory system,
a prohibited or indeed a permitted behavior, another affect such as
anger or arousal, a named identity, a script for interpreting other
people’s behavior toward oneself” (62)
\end{quote}
I'm interested in this move that she makes, of taking reading as a
process that is ultimately \emph{generative} and \emph{productive}.

This project takes this prospect quite literally: it uses text
generation as a reading practice. It takes what is typically seen as a
contentious, the concept of gendered embodiment between the
conservative and progressive poles, and sees how one might do
something productive within it.

\item the ML process aggregates perspectives on subjectivity
\label{sec:orgc3b94c9}
The reason we get these outputs for subjective, is that this
particular term, "subjective" does not reflect the conservative
position from the Heritage Foundation data. Rather, it reflects a
conservative frame for the progressive position. In other words, it
represents what a conservative thinks a progressive person thinks
gender is---as something insubstantial, as a feeling. In the outputs
then, we see not just a single perspective of gender, but a
\emph{flattening} or \emph{aggregation} of perspectives into a single statement.
The ML process underlying the language model takes these distinct
viewpoints and aggregates them into an apparently univocal utterance.

My project reads polarized perspectives to explore how they might
intersect. To explore what is held in common between polarized points
of view.

We will use this aggregative method to study cisgender experiences of
embodiment. What we will get at is a kind of cross section of what the
cis and the trans might have in common.
\end{enumerate}

\subsubsection{{\bfseries\sffamily TODO} TRANSITION \emph{LiB}'s transgressive premise}
\label{sec:org1195824}
The point of going on the show, as one participant puts it, "to be
loved for who I am on the inside,” (Season 2, Episode 3, "Love in
Paradise”). However, the show’s main gambit, that "love is blind,”
suggests a transgressive premise that undergirds an ultimately
heteronormative teleology. Despite the promise of marriage and
happily-ever-after, something non-normative is happening here. And
that something has to do with the body, and what happens to the body
when it falls in love from behind a wall.

This paper explores, in the words of Micah above, how "touch” can be,
for cis people, their "worst nightmare." I take readings of
normativity from Trans Studies and apply them to an analysis of
cisgendered subjects from the show. I examine what this situation,
where visual access to the beloved is denied, does to the
self-perception of the body of the participants. I find that this
"blind” dating experiment places participants in a state where their
own bodily coherence fractures, which has consequences on their
romantic trajectory and aspirations. While firmly anchored to their
cisgendered identities, the participants undergo a split in the
physical body, which begins to accrue investments to integrity and
wholeness that inevitably go unfulfilled once they are united with
their beloveds.

Throughout this process, I argue, they enter a version of what Jay
Prosser calls the "transsexual trajectory” (6). For Prosser, this
trajectory "bring[s] into view the materiality of the body," in
particular the internal "body image" that is distinct from and
contained within the external, physical body (12). For trans subjects,
the body image is related to feelings of bodily dissociation and
dysphoria, which is not the case for these cis subjects, who are
(apparently) anchored to their sex-gender identities thoughout the
show. Nonetheless, these subjects, sequestered from the sight of their
beloved, undergo a bifurcation within the body. The wall which
separates their embodied feelings from the vision of the other places
them, temporarility, on the "route to identity and bodily integrity"
(6).

To analyze this trajectory in LiB, I created a tool that surfaces the
participants speech about their bodies and their experiences of
embodiment. Using the transcripts from the show, which I scraped from
the internet,\footnote{To scrape the transcripts, I wrote a web crawler using the
\emph{scrapy} library in the Python programming language. This program
allowed me to "crawl," or paginate through, the transcript episodes
stored on this website:
\url{https://subslikescript.com/series/Love\_Is\_Blind-11704040}. The code
used to create the crawler and the transcripts themselves are saved on
my github account, under the "gofilipa/love\_blind repository," (see
Calado, "love\_blind".} I built a text generator that mimics the speech of
the participants. I use this text generator, which synthesizes common
patterns and shared investments in language (a process I explain
below), to analyze characters' perspectives of embodiment throughout
the show, from their experience in the pods up until the wedding day.

\subsubsection{{\bfseries\sffamily TODO} ADD citations to recent reporting on AI}
\label{sec:orgf3e6ce0}
Although this method makes use of machine learning (ML) technology, it
does so deliberately to resist dominant uses of that technology today,
particularly what Gael Varoquaux et al. describe as the
"bigger-is-better mentality" that drives ML development.\footnote{I do not use the term "AI," but machine learning.} This
mentality has to do with the belief that more data (scraped from the
internet) and more "compute" (Graphical Processing Units, or GPUs,
sourced from deep Earth minerals) will lead to better performing
models.

The drive for larger models has spurred more and more investment,
which has inflated the economy to what many project are
bubble-bursting levels (CITATION), as many tech companies like OpenAI
are running on pure investment and do not project to be actually
profiting from their product for several years (CITATION).
Additionally, as recent research points out, this bigger is better
drive is counter-intuitive: Large Language Models actually have a
ceiling in terms of size and how it affects performance; that
ever-increasing compute does not yield comparable returns in terms of
the quality of model outputs (CITE Varoquaux and another?). Which
makes the tech companies all the more desperate to protect their
investments at all costs. Perhaps why OpenAI is lowering safeguards
for ChatGPT, allowing it to create porn (CITATION).

Together, general ignorance about so-called "AI" and market incentives
combine to fuel what Emily Bender and Alex Hanna have usefully termed
"AI hype" (\emph{The AI Con}). Hype is a self-reinforcing and perpetuating
mechanism driven by ignorance about how models actually operate (the
race after "intelligence") and capital's desperation for profit above
all else. This creates a self-perpetuating pressure to adopt the tool
that infects every level of the workplace: CEOs feel it, managers feel
it, mid-level and entry-level workers feel it.

What I offer, by contrast, is a methodology that rejects this high
consumption mentality, opting instead for small models and datasets,
and for deliberate attention to the mechanism's operation under the
hood. The LLMs that I use for this project, which I jokingly call
"small language models," were trained on a single laptop, over a
single afternoon.\footnote{For example, this project uses GPT-2, which is intially trained off
only 8 million webpages and released under an open license. Compare
that with the most recent version of GPT, GPT-5, which is trained on
something like the entire internet and is over 600 billion parameters
in size, a number that cannot be confirmed due to its closed and
proprietary status.} The dataset which I used for training (more on this
process below) was similarly small in size, containing the transcripts
from one season, 13 total episodes, of the show.\footnote{The code, training data, and models developed for this project
are openly licensed on the Github and Huggingface platforms. I use the
GPL 3.0 license, which allows users to freely run, modify, and
distribute the project while ensuring that all modified versions
remain free as well. (See Calado, \emph{love\_blind} code repository and
Calado, \emph{LoveIsBlind\_Pods} and \emph{LoveIsBlind\_Postpods} model
repositories.)

. published under \url{https://github.com/gofilipa/love\_blind} and
the models are published under \url{https://huggingface.co/gofilipa}}

My analytical approach turns on the specifics of the prediction
process, as I explain below. The current emphasis on using machine
learning as a tool for productivity, to generate \emph{new} content, while
serving extractive and monetizing purposes, misses the fact that these
tools are primarily \emph{self-reflexive}. As Wendy Chun points out,
predictive tools are good for studying existing patterns in data, and
might be used to study patterns so that they can be resisted or
avoided. Her work, which carefully traces the racist, specifically
eugenicist, origins of statistical processes,\footnote{Include some of this eugenicist history of stats tools. Can be
brief.} which forms the
foundation of machine learning technology today, proposes that these
tools be used for revealing patterns that are harmful or otherwise
undesirable, so that one might act differently. She offers the example
of one area which already does this work, climate change modelling.
Here, she asks: "How can we treat machine learning systems and their
predictions like those for global climate change? These models offer
us the most probable future given past and current actions, not so
that we will accept their predictions are inevitable, but rather so we
will use them to help change the future" (26). My approach takes a
similar shift around machine learning, using it as a tool for creating
new objects toward "productivity", but to better understand the data
that we already have.

Prediction, in addition to being a descriptive mechanism, is also a
normalizing one. Within the prediction process itself, protocols find
and amplify \emph{frequent patterns} of word usage. This mechanism of
amplifying what is frequent or common in language data distils the
dominant tendencies and perspectives within show transcripts into the
generated outputs. The predictions, then, will represent an
approximation of what is most typical or natural in training data. As
a kind of normalizing mechansism, then, Prediction, is an apt tool for
studying shared desires---in my case, with the \emph{LiB} subjects, for
studying a shared desire for marriage.

Normativity, and desire for what Andrea Long Chu describes as "a
normal fucking life" is one place where trans subjects intersect with
with cisgendered subjects (Chu and Drager 107). The element of
normativity is where many Trans Studies scholars differentiate Trans
Studies from Queer Studies. Jay Prosser, for for example, in his
seminal work \emph{Second Skins: The Body Narratives of Transsexuality}
(1998), explains that trans stories seek embodied normativity as a
\emph{telos}. Prosser's argument, which I examine in more depth below,
reads how the experience of embodiment factors in a material way into
the stories of trans subjects, which he calls "transsexual
trajectories". While queer subjects move against "the apparent
naturalness of sex," trans subjects, by contrast, "seek very pointedly
to be nonperformative, to be constative, quite simply, \emph{to be}" (32).

\subsubsection{the method}
\label{sec:org5bbfde4}
I will now demonstrate how this normalizing mechanism works in depth,
using examples from the model that I trained from the transcripts of
the show. I prompted this model with the phrase "Marriage is." It then
generated the following outputs:
\begin{quote}
Marriage is not an easy decision.

Marriage is not a celebration.

Marriage is a lifelong commitment. (Appendix 2: Postpod prompts)
\end{quote}
None of these sentences appear in the transcripts of the show. Rather,
the second half of the sentences in these outputs are filled in by
phrases that appear in \emph{similar contexts} with the word "Marriage" in
the transcripts. Instead of reproducing verbatim expressions, the
model generates approximations of expressions within the transcripts.
These approximations are a result of calculations, a series of
statistical calculations, which determine the word that is most likely
to appear next.

In order to understand how this statistical calculation works, one
must first understand what happens to words themselves, and how word
meaning is represented within the model. The model represents words in
a numerical form, which is technically called a "word vector." Word
vectors are how a machine learning knows what words mean individually,
they comprise the model's internal dictionary, so to speak. The
vectors themselves consist of a large and complex list of numbers,
representing probablitity scores. Each of the numbers in the vector
represent's a given word's association to another word in the dataset.
For example, "marriage" may have a high probability score with the
word "committment," and a lower probability score with the word
"apple." Once a word is quantified into a word vector, its meaning can
be then be calculated by the machine. One can do math with words.

In order to compile word vectors, however, the model must first be
trained on a dataset, such as the transcripts of the show. There are
three steps to the training process: their technical names are (1)
hypothesis, (2) loss, and (3) optimization.

First, in the hypothesis step, the model takes a sample sentence from
the transcript, like "marriage is not easy" and it blocks out the
second half of the sentence, so that only "marriage is" remains (\emph{Love
Is Blind}, Season 2, Episode 14). It tries to guess which word or
phrase should go in that second half, "Marriage is an apple." Moving
to the second step, loss, it checks its prediction against the actual
sentence, "Marriage is not easy." In this case, loss represents the
mathematical difference between the vector for "not easy" and the
vector for "apple." Then, it moves to the final step, optimization.
Here, the model uses an algorithm to calculate the smallest adjustment
possible that it can make to the vectors so that they are just
slightly closer to the actual result. The adjustment must be
miniscule, but it is precise. At each training step, the model slowly
closes the gap between the prediction and the actual result. 

The model will repeat these three steps over and over, making guess
after guess after guess. It will try out many words, perhaps every
word in the dataset, until it is sure of those that are most likely to
appear together. With each guess, the model makes very slight
adjustments to its own representation of word meaning (this constant
iteration, and the computer processing required to do it, is why
language models take lots of time, energy, and computer hardware to
train). By the end, the list of probabilities will reflect a kind of
average of that word's association to other words.

For example, in the prompt, "marriage is," the model will ascertain
possible completions for this phrase, given other words that are
associated with "marriage" in the dataset. One actual completion it
gives, "not easy," reflects an implicit association between "marriage"
and committment. In the show transcript, the phrase appears during the
period of the show when the couples are living together, prior to the
wedding. Here, one participant, Jarrette, describes his difficulty
adjusting his lifestyle to the new committment: 
\begin{quote}
Marriage is not easy. Over the past couple of months, like, I've
definitely been struggling with coming in late, um, and just
overindulging when I'm out. I haven't been the best at prioritizing
us. And, uh, it got to a point where Iyanna moved out. Season 2,
Episode 14
\end{quote}
This context influences the model's interpretation of the word
"marriage." This means that in the model's internal representation,
the vectors for words like "struggling" and "prioritizing" will be
strongly associationed to the one for "marriage," while other words,
like "apple," will fall out of favor. The effect is that when
prompted, the model will generate completions like,
\begin{quote}
Marriage is not an easy decision.

Marriage is not a celebration.

Marriage is a lifelong commitment. (Appendix 2: Postpod prompts)
\end{quote}
These completions are not exact, verbatim examples from the show
transcripts: "Marriage is not easy" is slighly different from
"Marriage is not an easy decision." Generating outputs that exactly
resemble the training data is undesirable model behavior (technically
called "overfitting," which I discuss in detail below). The goal,
rather is to generate \emph{plausible} outputs, given the context of the
training data. I read this guessing mechanism, which approximates word
meaning from a variety of samples, as a kind of normalization of
language. The model generates language by approximating what is most
likely, most plausible, based on its training data. As such, it is
ideal for studying shared or common among the participant experiences
on the show.

\subsubsection{the outputs}
\label{sec:org426b6a2}
The show contains two stages, the pre-engagement stage, where
participants date each other from separate "pods" where they can hear
but not see the other, and the engagement stage, where participants
finally meet and proceed to live together in preparation for the
wedding. In partitioning the romantic experiment into pre-engagement
and engagement segments, the show poses the presence and role of the
body as the variable that ultimately determines the viability of
long-term commitment. In other words, it sets up an examination of how
the body may affect normative trajectories and desires.

To explore the differences between these periods of the show, I
trained two separate models on each of these periods: the first model
is trained on the transcripts from the pods episodes, and I call it
the "pods model"; the second model is trained on the episodes
following the pods, and I call it the "postpods model."

Then, I prompted both models with input phrases about the body and
touch to see how each of them would respond individually to the
prompts. When prompted with phrases like "When I touch you," and
"Physical touch is," and "Physically," the pods model generates the
following outputs:

\begin{quote}
When I touch you, I feel it.

When I touch you, I feel your energy, and it is, like, I'm just so
happy.

When I touch you, I can feel your soul, your heart, and your soul
aligning so well.

When I touch you, I feel you, and I feel you, I feel you, and I feel
you.

Physical touch is important to me.

Physical touch is the most important thing.

Physical touch is so sexy.

Physical touch is like a glove.

Physically, we are so happy.
\end{quote}
The model's prediction mechanism can create all kinds of quirks in the
outputs. Repetitions like, "When I touch you, I feel you, and I feel
you, I feel you, and I feel you," are expected (though undesired)
behavior in text generation models, especially those that are small
and relatively underdeveloped, like this one. Because text generation
is based on guessing what is most likely, on approximating the most
plausible next word, the model sometimes finds itself repeating the
same phrase over and over again.

While models are good at prediction, they are not at all good
at being creative, at innovating. A model can only generate what it
has already seen before. Even a phenomenon like "hallucination," that
a model spews text that has no bearing in reality, is based on the
tendency of models to repeat what they’ve already seen. They
hallucinate not because they are creative or random, but because they
are designed from statistical processes to produce what is most
plausible rather than most accurate.

The rest of the results, then, reflect what is most plausible given
the information from the the transcripts. Because these transcripts
are from a period of the show when no actual touching occurs between
the couples, the model associates touch with non-tangible phenomena,
like "soul" and "energy." In addition, touch---being foreclosed from
the participants during this stage of the experiment--is elevated as
something highly desired, to an "important," "most important," and
even "sexy" quality. Finally, the last two examples, "like a glove"
and "we are so happy," suggest an association between touch and
compatibility, in the sense that the couples "fit" together, so to
speak.

The outputs with those from the postpods model, however, put touch in
very different contexts:

\begin{quote}
When I touch you, I feel like I'm in my head.

When I touch you, like, I feel like I'm literally in my head.

When I touch you, you just feel like it's so weird.

When I touch you, it feels like a jab.

When I touch you, it feels like something I'm about to get up and
walk away.

When I touch you, I feel like it's like I've just, like, left the
room.

When I touch you, the thing that's scary is, like, it's a physical
thing.

When I touch you, you're like "I'm blinking."
\end{quote}
Unlike the pods model, these outputs present touch as a not pleasant
experience. Touch is "so weird," "like a jab," suggesting a strange
and even disruptive aspect to touch. Furthermore, touch is associated
with physicality, which is "scary." At the same time, touch signals
location and movement: "in my head," "walk away," and "left the room."
While in the pods, touch drew the characters together, evoking
non-tangible phenomena like the soul and energy, here it seems that
touch repels the characters from each other.

Turning to the transcripts can help to inform some of these outputs.
For example, the concept of touch being scary emerges in one scene
between Paul and Micah, a newly engaged couple who are one day into
their pre-wedding romantic getaway, in Mexico. They are swimming in a
freshwater pool, when Micah notices a fish, which she mistakes for a
small shark,
\begin{quote}
Micah: Is that a shark?

Paul: What do you mean shark? Oh, they're catfish. They look like
baby, like, tiny nurse sharks.

Micah: Okay, this is kind of scaring me. It's my worst nightmare if
one touches me.

Paul: These little catfish?

Micah: Yeah.

Paul: You're not their mommy.

\emph{Love is Blind}, Season 4, Episode 5, "Paradise Lost"
\end{quote}
Here, the word "touch," which appears in the same sentence as "my
worst nightmare," accrues association with fear and revulsion. It is
also, interestingly, associated with the concept of motherhood, in the
statement from Paul, "You're not their mommy." At the end of the show,
Paul rejects Micah directly on the altar. The reason, he later
explains, is because he "struggled with\ldots{} envisioning Micah as, like,
you know, a mother" (Season 4, Episode 12, "Eternal Bliss?").

Most of these outputs represent approximations, but also direct quotes
taken from the transcripts. The phrase that says, "I'm blinking," is
actually taken directly from the show, and is an example of an
undesired but not uncommon blip in the prediction process. In machine
learning, this blip is referred to as "overfitting," when a verbatim
section of text from the training data, in this case, the show
transcripts, is generated in the output. "Overfitting" means that the
model is too accurate: that it has slipped from making predictions
that are plausible to repeating exactly the data it has been trained
on. A model overfitting in its outputs is generally a sign that there
isn't enough training data or enough variation in the training data,
meaning that the model has less examples from which to approximate and
generalize. So, it resorts to simply reproducing direct examples from
its training.

For my purposes, however, overfitting is not only a blip, it also
points to a specific scene in the show, which highlights a tension
between the sensory modes of touch and sight. The original reference
to "blinking" appears in a scene with the newly engaged couple, Zach
and Irina, when they meet each other for the first time in person. The
doors open, and they awkwardly approach each other down a red carpet.
After exchanging their first greetings, they have a conversation about
their reaction to each other's appeareance:
\begin{quote}
Zach: Do I look like what you thought I'd look like?

Irina: I had no guesses of what you looked like.

Zach: Oh!

Irina: You have, like, the blankest stare in your eyes.

Zach: Really?

Irina: I'm just kind of taking it all in.

Zach: Me too.

Irina: You look like a fictional character. You look like something
out of a cartoon.

Zach: I know.

Irina: You have to blink!

Zach: I am blinking.

Irina: You don't blink. You look like this.

Zach: I am blinking. I will try not to be too intense. (Season 4,
Episode 4, "Playing with Fire")
\end{quote}
Zach seems a bit insecure of his appearence, asking if he looks how
Irina imagined. And Irina, in turn, seems put out, describing him as a
"fictional character" and demanding that he blink. Blinking is, of
course, a way of stopping the entry of visual data, of occluding it
from perception. For Irina, the request for Zach to blink might
indicate her own sense of overwhelm at his physical form, at his
sudden incorporation before her eyes. Perhaps, the reality of his
physical form is too much, so that, projecting her own feelings of
overstimulation, she asks him to blink.

In the story of Zach and Irina relationship, it is clear that the
catalyst for their breakup is a lack of physical attraction on the
part of Irina. Later in the same episode, Irina explains her feelings
to Micah, the same Micah from the "shark" scene, who is coupled with
Paul. 
\begin{quote}
Irina: And so, Zack. I feel like is my type on paper. Has, like, brown
hair, brown eyes, like, chiseled face. Like, I really like dark
features. And the moment I saw Zack, it was like, "I don't know who
this man is." And I was like, "Maybe it's just scary, and it was a
lot." Like, hopefully it's gonna grow, but I've noticed every time he
does, like, touch me, I get, like, major ick. When he puts his arm
around me at night, I literally was like-- like, my heart stopped. And
I literally go\ldots{} But not, like, in an excited way.

Micah: I wanna, like, relate to you in a way, but it's always, like,
so different.

Irina: How was it with you and Paul?

Micah: The thing with me and Paul is, like, we both, like, had such an
immediate understanding as best friends.

Irina: Yeah, Paul's gorgeous. (Season 4, Episode 4, "Playing With
Fire")
\end{quote}

Zach supposedly has physical aspects which Irina finds attractive,
"brown eyes, chiseled face," but something about him nonetheless
repulses her. When he puts his arm around her, she recoils,
"get[ting]… major ick." While she claims her feeling of disgust have
nothing to do with his physical appearance, Irina simultaneously
conjures appearance with the phrase, "Yeah, Paul's gorgeous." To
Irina, Paul's "gorgeous[ness]" explains why Micah and Pual had an
"immediate understanding."

I want to propose that Irina's physical repulsion to Zach results from
the experience in the pods, from the forclosure of the visual sense
within the pods. Irina asks Zach to "blink" in attempt to relieve her
own feeling of being overwhelmed by the sight of him. This sensual
overload related to sight also relates to touch: she is repulsed by
his physical touch, his closeness; there is something about touch
which similarly overwhelms her. 

That is because, sequestered from the sight of the other within the
pods, the participants experience a kind of bodily split in which
their own physical senses materialize in new and surprising ways. They
experience not only the physical body, the material reality of their
physical body which they've always known, but something like what Jay
Prosser refers to as the "body image," an internal perception of the
body. Despite being internal, the body image is a physical, sensual
phenomenon, which "clearly has a material force for transsexuals,"
according to Prosser (69). For trans subjects, this "material force"
often manifests in the trope of being "trapped in the wrong body" and
feelings of dysphoria. The sensory deprivation of being in the pods, I
argue, subjects these cisgendered participants to something akin to
Prosser's bodily split, from which the body image emerges---for a
time.

For these subjects, I argue, the body image manifests in a heightened
sensation of the body, which paradoxically creates a feeling of the
body's dissolution. When prompted with the phrase "My body," the pods
model generates the following completions:
\begin{quote}
My body feels like it's coming off.

My body feels heavier.

My body feels so different now.

My body feels weird.

My body makes me feel like it's real.

My body feels torn between two different people.
\end{quote}
Across all the samples, there is an increased awareness of the
physical body. Due to the denial of visual access, the body's
physicality comes into apprehension in a novel and visceral way, which
makes it seem all the more strange. Perhaps due to the absence of the
visual sense, the body feels "weird" and "so different now." Weighted
by this new somatic "heavi[ness]," the body even "feels like it's
coming off," an image that evokes Prosser's concept of the bodily
split between the physical body and the body image. Perhaps, the
reference to being "torn between two different people" not only refers
to actual people, to dating multiple characters on the show, but to a
single person with two bodies in tension. As Prosser explains, the
body split creates "conflict\ldots{} between sentient body image and
insentient visible body" (70). For these straight, cisgendered
participants within the pods, the body image may be coming into
sentience in a way that is not possible when they are fully
integrated, outside the pods. And this may be the first (and only)
time that they experience this level of bodily sensation and
awareness.

But it is not a feeling that lasts long. In the postpods model, the
body appears to be re-integrated. The "insentient visible body," the
outer body, comes into view when the participants are finally given
visual access to each other. Here, the language about the body shifts
into notably more visual and positive descriptions:
\begin{quote}
My body is gorgeous.

My body is so cute.

My body is so pretty.

My body makes me feel lighter, more confident.

My body makes me feel warm.

My body makes me feel like I've missed my train.
\end{quote}
The outputs address the body in concise and flattering terms: the body
is "gorgeous," "so cute," "pretty." Now that the visual sense has been
re-incorporated to the body, it becomes the dominant sense modality.
Perhaps, because the couples can see each other, the body feels
"lighter" and "warm," offering coherence where before was weirdness
and weight. In the last output, however, there is a suggestion of
something not quite right: "My body makes me feel like I've missed my
train." This statement, with its slightly nostalgic undertone,
suggests that even when coherence is gained, something is lost.

What is lost comes into view when the model is prompted with
statements about touch, the sense modality that has been foresaken for
the visual:
\begin{quote}
Physical touch is everything that I've wanted in a wife.

Physical touch is everything that I've ever wanted in a partner.

Physical touch is a big part of what I want.

Physically, there's so much potential here.

Physically, it was the perfect opportunity.
\end{quote}
Physical touch is described in aspirational terms: it is "everything
i've wanted," "everything I've ever wanted," and "what I want." The
past perfect tense here, and the reference to unfulfilled opportunity
is indicative: even after meeting in person, the desire seems to
freeze in place. The restoration of the visual sense, the
re-integration the previously fractured body, then, does not offer
completion or culmination.

Being restored their visual sense heals the LiB participants from the
bodily split, but it does not save them from the aftermath of their
investments. When the couples finally meet in physical forms, they
remain plagued by the possibilities for physical connection that they
felt in the pods. And these expectations are what, for some of them,
prevents their ability to accept their partners as they are. Due to
their experience in the pods, the significance of touch is inflated to
include other, perhaps practically unattainable, desires. Although
they exited from one trajectory, they remain stuck with the desire for
a kind of touch that is "everything that I've ever wanted in a
partner" (Appendix 1). Considering that the characters are now
reunited with their physical bodies, there is something almost cruel
in this denouement, a "cruel optimism," in Lauren Berlant's
formulation, which describes the attachment that drives desire even
while it wears out the desirer.

Or, more specific to their bodily predicaments, the characters
experience a version of what Hil Malatino describes as "future
fatigue" (20). Like cruel optimism, future fatigue generates "intense
anticipatory anxiety" that "impede[s] flourishing" (Malatino 20).
Unlike cruel optimism, however, future fatigue concerns trans subjects
who are invested in "the promised moment of harmony between the felt
and the perceived body" (Malatino 27). Despite being cisgendered,
these subjects in LiB experience a bodily split, during which they
develop romantic feelings and attachments. And most of them, when they
leave the pods, cannot fulfil these aspirations within their embodied
lives.
\subsubsection{{\bfseries\sffamily TODO} ADD close reading: approximating average sensations}
\label{sec:org9dc2ff1}
\subsubsection{cis and trans futures}
\label{sec:org91a8664}
The disruption of bodily integrity has ramifications that last well
beyond the pods. Recalling the case of Paul and Micah, Paul eventually
rejects Micah because he wanted more evidence of maternal qualities.
In his own words, he "struggled with\ldots{} envisioning Micah as, like,
you know, a mother" (Season 4, Episode 12, "Eternal Bliss?"). At the
end of the season, when the host of the "Reunion" episode prods him on
the topic, he struggles to articulate his reasoning:
\begin{quote}
Paul: You know, she didn't feel comfortable with showing that side of
her.

Host: I thought she said she did. She just talked about it from since
the pods up until your wedding day.

Paul: It wasn't evident to me… That wasn't there.

Host: You wanted actions. Like, you wanted her\ldots{}

Paul: I just wanted to be able to see it, I guess. Like\ldots{}

Host: What would make you see that? I'm sorry. Just so I understand.

Paul: I think\ldots{} It's a little bit ineffable, right? So, it's kind of
an exuding, a nurturing presence. It's something that you feel.
There's not really, like\ldots{} tangible, kind of like, things, I don't
think. (Season 4, Episode 13  ‘The Reunion")
\end{quote}
Paul claims that he failed to "see" evidence of Micah’s motherly
nature, but when asked to explain what this evidence would look like,
answers that it is "ineffable." He supposedly wants someone who can be
a mother, but cannot describe what a mother looks or acts like. What
he wants Micah to have expressed, he cannot express himself. At the
end, he takes refuge in the sense of \emph{feeling}, in the gravious
untraceability of felt sensation. 

But, the reader will know, that feeling and touch, in particular, does
occupy a very real and traceable sense modality---one that is
emphasized especially when another, like sight, is diminished.
Clearly, Paul's answer demonstrates that some cis people, besides
being firmly anchored to their genders, are not okay in their bodies.
That being said, however, I am not interested in a critical analysis
that redeems them or their bodily experience. After all, cis-hetero
bodies are already well represented and redeemed. What I am interested
in, rather, is the study of normativity, and how alignments between
cis and trans experience might further this study.

Which brings me to my final point: that I intend for this critical
method, which approximates language as a means of surfacing shared
investments, to push back against the polarization that characterizes
the current discourse in the US, and specifically that about trans
rights. In this paper, I deliberately isolated the body as a potential
vector of connection that flows through and between gender and sexual
identities. I believe that machine learning, with its tendency to
amplify what is most frequent, might reveal something shared about the
body, even across very different embodied experiences.

\subsubsection{{\bfseries\sffamily TODO} ADD toward new solidarities}
\label{sec:org78d027c}
\begin{itemize}
\item this revelation might offer groundwork for new solidarities between
trans and cis subjects; solidarity based on the strange feeling of
"the body coming off"
\item bring back the reading of "reality?"
\end{itemize}

This is an attempt to expand Trans Studies by applying its interest in
the body beyond the proper or conventional areas and objects. And
Trans Studies offers no dearth of theorizing on the body and its
desires.  I close with one quote from a study by Cassius Adair that
explores "t4t erotics." Here, Adair asks, "Why can't the erotic be a
site of producing trans identity or practices?" (46). He points out
that, after all, "cis people do it all the time: all erotic desires
might be sites of identity formation, for anyone" (47).

Yes, cis people do it all the time. But what they tend not to do, and
which Trans Studies does so well, is to question the perimeters of
their own bodies, to explore how desire and attachments manifest in
material ways on the body. And perhaps, as I hope this paper has
shown, that Trans Studies theorizing might open up such thinking about
the body, "for anyone."



\subsubsection{bank}
\label{sec:org489e01f}
\begin{enumerate}
\item sedgwick quotes
\label{sec:org2aef028}
"for someone to have an unmystified, angry view of large and genuinely
systemic oppressions does not \emph{intrinsically} or \emph{necessarily} enjoin
on that person any specific train of epistemological or narrative
consequences. To know that the origin of HIV \emph{realisitically might}
have resultled form a state-assisted conspiracy---such knowledge is,
it turns out, separable from the question of whether the energies of a
given AIDS activist might best be used in the tracing and exposure of
such a possible plot. They might, but then again, they might not"
(\emph{Novel Gazing} 4).

\item homophily: negative feelings are a tool for polarization
\label{sec:orgce6552c}
Chun - Homophily: Negative feelings are a tool of polarization, a way
of aligning people around a cause, but in a way that also keeps them
separate. For example, incels. Explained via magnetic polarization:
"polarized filings both repel one another and stick together through
their overwhelming attraction to their opposite pole" (Chun 85).

\item introducing study of normativity in Trans Studies
\label{sec:org1bfd55f}
—-a crucial area of interest for Trans Studies, as Andrea Long Chu
puts it: "Trans Studies requires that we understand—as we never have
before—what it means to be attached to a norm, by desire, by habit, by
survival" ("After Trans Studies" 108).

\item focus on LARGE language models is reinforcing research patterns
\label{sec:orgc9c26d8}
"The bigger-is-better norm is also self-reinforcing, shaping the AI
research field by informing what kinds of research is incentivized,
which questions are asked (or remain unasked), as well as the
relationship between industrial and academic actors." (Varoquaux et
al). 

\item normativity quotes
\label{sec:org565a44a}

Prosser:

This desire normativity leads to a consideration of the body.
\begin{itemize}
\item Goal is to then read materiality into the transsexual body: "To
bring into view the materiality of the body" (Prosser 12).
\item Prosser: Critique of Queer Studies and gender performativity, that
it elides the specificity of the trans body.
\begin{itemize}
\item "\emph{Gender Trouble} cannot account for a transsexual desire for
sexed embodiment as \emph{telos}" (Prosser 33).
\begin{itemize}
\item gender crossing is essential to performativity, but not the
actual material body.
\end{itemize}
\end{itemize}
\end{itemize}

Queer v Trans subjects:
\begin{itemize}
\item "There is much about transsexuality that must remain irreconcilable
to queer: the specificity of transsexual experience; the importance
of flesh to self; the difference between sex and gender identity;
the desire to pass as "real-ly-gendered" in a world without trouble;
perhaps above all, as I explore in my next chapter, a particular
experience of the body that can't simply transcend (or
transubstantiate) the literal" (59).
\end{itemize}

And this attachment to normativity, in fact, is one way that trans
studies has distinguished itself with regard to queer studies, at
least according to some scholars.
\begin{quote}
“trans analytics have (historically, though not universally) a
different set of primary affects than queer theory. Both typically
take pain as a reference point, but then their affective interest
zags. Queer relishes the joy of subversion. Trans trades in quotidian
boredom. Queer has a celebratory tone. Trans speaks in sober detail.”
\end{quote}
\end{enumerate}

\subsection{Works Cited}
\label{sec:orgaeb7153}
Adair, Cassius, and Aren Aizura. "‘The Transgender Craze Seducing Our
[Sons]’; or, All the Trans Guys Are Just Dating Each Other." \emph{TSQ:
Transgender Studies Quarterly} 9.1 (2022): 44–64.

Bender, Emily M, and Alex Hanna. \emph{The AI Con: How to Fight Big Tech’s
Hype and Create the Future We Want}. New York, NY: Harper, an imprint
of HarperCollinsPublishers, 2025.

Berlant, Lauren Gail. \emph{Cruel Optimism}. Durham: Duke University
Press, 2011.

Calado, Filipa. \emph{anti-trans} code repository, \emph{Gofilipa}, Github.
\url{https://github.com/gofilipa/anti-trans}. 2025.

---. \emph{gpt2-hertiage\_foundation-gender} model repository.
Huggingface.
\url{https://huggingface.co/gofilipa/gpt2-hertiage\_foundation-gender}. 2025.

---. \emph{gpt2-aclu-gender} model repository. Huggingface.
\url{https://huggingface.co/gofilipa/gpt2-aclu-gender}. 2025.

---. \emph{love\_blind} code repository, \emph{Gofilipa}, Github.
\url{https://github.com/gofilipa/love\_blind}. 2025.

---. \emph{LoveIsBlind\_Pods} model repository. \emph{Gofilipa}, Huggingface.
\url{https://huggingface.co/gofilipa/LoveIsBlind\_Pods}. 2025.

---. \emph{LoveIsBlind\_Postpods} model repository. \emph{Gofilipa}, Huggingface.
\url{https://huggingface.co/gofilipa/LoveIsBlind\_Postpods}.
\begin{enumerate}
\item 
\end{enumerate}

Chu, Andrea Long, and Emmett Harsin Drager. "After Trans Studies."
TSQ : \emph{Transgender Studies Quarterly} 6.1 (2019): 103–116. Web.

Chun, Wendy Hui Kyong. \emph{Discriminating Data: Correlation,
Neighborhoods, and the New Politics of Recognition}. Cambridge,
Massachusetts: The MIT Press, 2021.

\emph{Love Is Blind}. Seasons 1-4, and 6. Netflix. 2020 - 2025.

"Love Is Blind (2020–\ldots{}) - episodes with scripts." Subs Like
Script. 2025. \url{https://subslikescript.com/series/Love\_Is\_Blind-11704040}

Malatino, Hil. \emph{Side Affects: On Being Trans and Feeling Bad}.
Minneapolis, MN: University of Minnesota Press, 2022.

Prosser, Jay. \emph{Second Skins: The Body Narratives of Transsexuality}.
Columbia University Press. 1998.


\subsection{Appendix 1: Pods Model Outputs}
\label{sec:org2dbc862}

Prompt: "Love is"

Outputs: 

"Love is really important to me."

"Love is what I need in my life."

"Love is everything that I've been looking for."

Prompt: "When I touch you"

Outputs:

"When I touch you, I feel it."

"When I touch you, I feel your energy, and it is, like, I'm just so
happy."

"When I touch you, I can feel your soul, your heart, and your soul
aligning so well."

"When I touch you, I feel you, and I feel you, I feel you, and I feel
you."

Prompt: "Physical touch is"

Outputs:

"Physical touch is important to me."

"Physical touch is the most important thing."

"Physical touch is so sexy."

"Physical touch is like a glove."

Prompt: "Physically,"

Outputs:

"Physically, we are so happy."

Prompt: "My body"

Outputs:

"My body feels like it's coming off."

"My body feels heavier."

"My body feels so different now."

"My body feels weird."

"My body makes me feel like it's real."

"My body feels torn between two different people."

\subsection{Appendix 2: Postpods Model Outputs}
\label{sec:org3182f9a}

Prompt: "When I touch you"

Outputs:

"When I touch you, I feel like I'm in my head."

"When I touch you, you just feel like it's so weird."

"When I touch you, it feels like a jab."

"When I touch you, like, I feel like I'm literally in my head."

"When I touch you, it feels like something I'm about to get up and
walk away."

"When I touch you, I feel like it's like I've just, like, left the
room."

"When I touch you, the thing that's scary is, like, it's a physical
thing."

"When I touch you, you're like 'I'm blinking.'"

Prompt: "My body"

Outputs:

"My body is gorgeous."

"My body is so cute."

"My body is so pretty."

"My body makes me feel lighter, more confident."

"My body makes me feel warm."

"My body makes me feel like I've missed my train."

Prompt: "Physical touch"

Outputs:

"Physical touch is everything that I've ever wanted in a partner."

"Physical touch is everything that I've wanted in a wife."

"Physical touch is a big part of what I want."

"Physically, there's so much potential here."

"Physically, it was the perfect opportunity."
\end{document}
